\section{Algorithmen}

    
    \begin{frame}
    \frametitle{Algorithmus für safety-H Methode (1)}
    Input: vollständige, minimale Referenz-DFSM $Ref_min$ und dazugehörige minimierte Abstraktions-DFSM\\
    Output: safety-Testsuite
    \begin{enumerate}
      \item Füge $V.\bigcup\limits_{i=1}^{m-n+1}.\Sigma_I^i$ der Testsuite hinzu
      \item Generiere Menge $A$, $B$ und $C$ aus der vollständigen, minimalen Referenz-DFSM
      \item Generiere $(\alpha, \beta)$-Liste aus $A$
      \item Iteriere über Liste:
      \begin{itemize}
        \item Berechne Zustand $q = \delta(q_0,\alpha)$ und $q'=\delta(q_0,\beta)$ in der \textbf{Referenzmaschine}
        \item Falls $q = q'$: Wähle $\omega$ und füge $\alpha.\omega$ sowie $\beta.\omega$ der Testsuite hinzu
      \end{itemize}
    \end{enumerate}
    \end{frame}
    
    \begin{frame}
    \frametitle{Algorithmus für safety-H Methode (1)}
    \begin{enumerate}
      \setcounter{enumi}{4}
      \item Generiere neue $(\alpha, \beta)$-Liste aus $B$ und $C$
      \item Iteriere über Liste:
      \begin{itemize}
        \item Berechne Zustand $q = \delta(q_0,\alpha)$ und $q'=\delta(q_0,\beta)$ in der \textbf{Abstraktionsmaschine}
        \item Falls $q = q'$: Wähle $\omega$ und füge $\alpha.\omega$ sowie $\beta.\omega$ der Testsuite hinzu
      \end{itemize}
    \end{enumerate}
    \end{frame}
    
    \begin{frame}
    \frametitle{Algorithmus für safety-H Methode (2)}
    Input: vollständige, minimale Referenz-DFSM $Ref_{min}$ und dazugehörige minimierte Abstraktions-DFSM\\
    Output: safety-Testsuite
    \begin{enumerate}
      \item Iteriere über $(\alpha,\beta)$-Liste und erzeuge $TS_1$ wie in (1)
      \item Iteriere über $(\alpha,\beta)$-Liste rückwärts (zufällige Reihenfolge) und erzeuge $TS_2$
      \item Starte ALgorithmus neu mit $TS_1 \cap TS_2 (\cap TS_i)$
    \end{enumerate}
    \end{frame}
    
    \begin{frame}
    \frametitle{Minimale Testsuite?}
    \begin{itemize}
      \item Minimale Testsuite zu finden ist ein NP-vollständiges Problem
      \item Reihenfolge der Bearbeitung beeinflusst das Ergebnis
      \item Wegen veränderter Reihenfolge kann $TS_s$ größer werden als $TS$
    \end{itemize}
    \end{frame}
    
    \begin{frame}
    \frametitle{Algorithmus für safety-H Methode (3)}
    Input: vollständige, minimale Referenz-DFSM $Ref_{min}$ und dazugehörige minimierte Abstraktions-DFSM\\
    Output: safety-Testsuite
    \begin{enumerate}
      \item Führe die $H$-Methode durch und merke, welcher Test durch welches $(\alpha,\beta)$-Paar zustandegekommen ist, und ob dieser wegen $A$ hinzugefügt wurde.
      \item Iteriere über Tests
      \begin{enumerate}
        \item Iteriere über alle $(\alpha,\beta)$-Paar für diesen Test\\
        \begin{itemize}
          \item[] Falls $(\alpha,\beta)$-Paar wegen A erzeugt wurde: Nächstes Paar
          \item[] Falls für alle gilt $\delta(q_0,\alpha) = \delta(q_0,\beta)$ \textbf{in der Abstraktionsmaschine}
          \item[] $\Rightarrow$ Test kann entfernt werden
    
        \end{itemize}
      \end{enumerate}
    \end{enumerate}
    \end{frame}
    