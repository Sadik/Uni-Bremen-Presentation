\section{Evaluation}

\begin{frame}
  \frametitle{Fasten-Seat-Belt}
  \begin{table}[]
    \caption{Ergebnisse für die \glqq Fasten-Seat-Belt\grqq-DFSM.}
    \label{tab:fsbrtsx}
    \centering
    \begin{tabular}{|ccccc|}
    \hline
          & \#Testfälle & \#Testschritte & ms & $\text{RED}_H$\\ \hline\hline
        $\text{s-H}_1$ & 337         & 1487    &    23   &  31\% \\ \hline
    $\text{s-H}_2$ & 423         & 2835    &    285  & 34\%  \\ \hline
    $\text{s-H}_3$ & 399         & 2558    &    372   &  36\% \\ \hline
    \end{tabular}
\end{table}
\end{frame} 

\begin{frame}
  \frametitle{Statistische Experimente}
  \begin{itemize}
    \item Konstruktion von Referenzmodell und Abstraktionsmodell.
  \end{itemize}
  \begin{table}[]
    \centering
    \begin{tabular}{|l|l|l|l|l|}
    \hline
    \textbf{$|Abs_{min}|$} & \textbf{$|Ref_{min}|$} & \textbf{$\Sigma_I$}   & \textbf{$\Sigma_O$}  & \textbf{$m-n$}   \\ \hline
    [2--99]     & 100        & 4      & 8      & 0 \\ \hline
    4          & [10--100]   & 4      & 8      & 0 \\ \hline
    4          & 20         & [2--10] & 8      & 0 \\ \hline
    4          & 20         & 4      & [3--20] & 0 \\ \hline
    4          & 20         & 4      & 8      & [0--3] \\ \hline
    \end{tabular}
    \caption{Versuchsaufbau für die statistischen Experimente. 100 Durchläufe pro Experiment.}
\end{table}
\end{frame}


\begin{frame}
  \frametitle{Konstruktion 1 (modulo)}
  \begin{itemize}
    \item Konstruiere Referenzmodell $M$ mit $n$ Zuständen, Abstraktionsmodell $M'$ mit $m$ Zuständen.
    \item Nummeriere Zustände.
    \item Teile die Zustände des Referenzmodells in $m$ Äquivalenzklassen ein ($q_n \mod m$).
    \item Wähle $0$ als sicherheitskritische Ausgabe, alle anderen Ausgaben sicherheitsäquivalent zueinander.
    \item Belege leere Transitionstabelle mit zufälligen Werten.
    \item Füge nach jeder Belegung notwendige Implikationen durch.
  \end{itemize}
\end{frame}


\begin{frame}
  \frametitle{Konstruktion 1 (modulo)}
  \centering
  \begin{tabular}{cc}
    \begin{tabular}{|l|l|l|l|l|}
      \hline
      & \multicolumn{2}{l|}{I2O} & \multicolumn{2}{l|}{I2P} \\\hline
    q & a           & b          & a           & b          \\\hline\hline
    \cellcolor[HTML]{DAE8FC}0  &             &            &             &            \\\hline
    1 &             &            &             &            \\\hline
    \cellcolor[HTML]{DAE8FC}2 &             &            &             &            \\\hline
    3 &             &            &             &            \\\hline
    \cellcolor[HTML]{DAE8FC}4 &             &            &             &            \\\hline
    5 &             &            &             &            \\\hline
    \cellcolor[HTML]{DAE8FC}6 &             &            &             &            \\\hline
    7 &             &            &             &            \\\hline
    \cellcolor[HTML]{DAE8FC}8 &             &            &             &            \\\hline
    9 &             &            &             &           \\\hline
    \end{tabular}
    &
    \begin{tabular}{|r|r|r|r|r|}
      \hline
      & \multicolumn{2}{l|}{I2O} & \multicolumn{2}{l|}{I2P} \\\hline
    q & a           & b          & a           & b          \\\hline\hline
    \cellcolor[HTML]{DAE8FC}$0:=\{0,2,4,6,8\}$  &             &            &             &            \\\hline
    $1:=\{1,3,5,7,9\}$ &             &            &             &            \\\hline
    \end{tabular}
    \end{tabular}
\end{frame}

\begin{frame}
  \frametitle{Konstruktion 1 (modulo)}
  \centering
  \begin{tabular}{cc}
    \begin{tabular}{|l|l|l|l|l|}
      \hline
      & \multicolumn{2}{l|}{I2O} & \multicolumn{2}{l|}{I2P} \\\hline
    q & a           & b          & a           & b          \\\hline\hline
    \cellcolor[HTML]{DAE8FC}0  &  1           &            & 4            &            \\\hline
    1 &             &            &             &            \\\hline
    \cellcolor[HTML]{DAE8FC}2 &             &            &             &            \\\hline
    3 &             &            &             &            \\\hline
    \cellcolor[HTML]{DAE8FC}4 &             &            &             &            \\\hline
    5 &             &            &             &            \\\hline
    \cellcolor[HTML]{DAE8FC}6 &             &            &             &            \\\hline
    7 &             &            &             &            \\\hline
    \cellcolor[HTML]{DAE8FC}8 &             &            &             &            \\\hline
    9 &             &            &             &           \\\hline
    \end{tabular}
    &
    \begin{tabular}{|r|r|r|r|r|}
      \hline
      & \multicolumn{2}{l|}{I2O} & \multicolumn{2}{l|}{I2P} \\\hline
    q & a           & b          & a           & b          \\\hline\hline
    \cellcolor[HTML]{DAE8FC}$0:=\{0,2,4,6,8\}$  &             &            &             &            \\\hline
    $1:=\{1,3,5,7,9\}$ &             &            &             &            \\\hline
    \end{tabular}
    \end{tabular}
\end{frame}

\begin{frame}
  \frametitle{Konstruktion 1 (modulo)}
  \centering
  \begin{tabular}{cc}
    \begin{tabular}{|l|l|l|l|l|}
      \hline
      & \multicolumn{2}{l|}{I2O} & \multicolumn{2}{l|}{I2P} \\\hline
    q & a           & b          & a           & b          \\\hline\hline
    \cellcolor[HTML]{DAE8FC}0  & 1            &            &  4           &            \\\hline
    1 &             &            &             &            \\\hline
    \cellcolor[HTML]{DAE8FC}2 &  3           &            & 2            &            \\\hline
    3 &             &            &             &            \\\hline
    \cellcolor[HTML]{DAE8FC}4 &  1           &            & 2            &            \\\hline
    5 &             &            &             &            \\\hline
    \cellcolor[HTML]{DAE8FC}6 &  2           &            & 8            &            \\\hline
    7 &             &            &             &            \\\hline
    \cellcolor[HTML]{DAE8FC}8 &  2           &            & 0            &            \\\hline
    9 &             &            &             &           \\\hline
    \end{tabular}
    &
    \begin{tabular}{|r|r|r|r|r|}
      \hline
      & \multicolumn{2}{l|}{I2O} & \multicolumn{2}{l|}{I2P} \\\hline
    q & a           & b          & a           & b          \\\hline\hline
    \cellcolor[HTML]{DAE8FC}$0:=\{0,2,4,6,8\}$  & Y            &           & $0$            &            \\\hline
    $1:=\{1,3,5,7,9\}$ &             &            &             &            \\\hline
    \end{tabular}
    \end{tabular}
\end{frame}

\begin{frame}
  \frametitle{Konstruktion 1 (modulo)}
  \centering
  \begin{tabular}{cc}
    \begin{tabular}{|l|l|l|l|l|}
      \hline
      & \multicolumn{2}{l|}{I2O} & \multicolumn{2}{l|}{I2P} \\\hline
    q & a           & b          & a           & b          \\\hline\hline
    \cellcolor[HTML]{DAE8FC}0  & 1            &  0          &  4           &  3          \\\hline
    1 &             &            &             &            \\\hline
    \cellcolor[HTML]{DAE8FC}2 &  3           &            & 2            &            \\\hline
    3 &             &            &             &            \\\hline
    \cellcolor[HTML]{DAE8FC}4 &  1           &            & 2            &            \\\hline
    5 &             &            &             &            \\\hline
    \cellcolor[HTML]{DAE8FC}6 &  2           &            & 8            &            \\\hline
    7 &             &            &             &            \\\hline
    \cellcolor[HTML]{DAE8FC}8 &  2           &            & 0            &            \\\hline
    9 &             &            &             &           \\\hline
    \end{tabular}
    &
    \begin{tabular}{|r|r|r|r|r|}
      \hline
      & \multicolumn{2}{l|}{I2O} & \multicolumn{2}{l|}{I2P} \\\hline
    q & a           & b          & a           & b          \\\hline\hline
    \cellcolor[HTML]{DAE8FC}$0:=\{0,2,4,6,8\}$  & Y            &           & $0$            &            \\\hline
    $1:=\{1,3,5,7,9\}$ &             &            &             &            \\\hline
    \end{tabular}
    \end{tabular}
\end{frame}


\begin{frame}
  \frametitle{Konstruktion 1 (modulo)}
  \centering
  \begin{tabular}{cc}
    \begin{tabular}{|l|l|l|l|l|}
      \hline
      & \multicolumn{2}{l|}{I2O} & \multicolumn{2}{l|}{I2P} \\\hline
    q & a           & b          & a           & b          \\\hline\hline
    \cellcolor[HTML]{DAE8FC}0  & 1            &  0          &  4           &  3          \\\hline
    1 &             &            &             &            \\\hline
    \cellcolor[HTML]{DAE8FC}2 &  3           & 0           & 2            &  1          \\\hline
    3 &             &            &             &            \\\hline
    \cellcolor[HTML]{DAE8FC}4 &  1           & 0           & 2            &  7          \\\hline
    5 &             &            &             &            \\\hline
    \cellcolor[HTML]{DAE8FC}6 &  2           & 0           & 8            &  3          \\\hline
    7 &             &            &             &            \\\hline
    \cellcolor[HTML]{DAE8FC}8 &  2           & 0           & 0            &  3          \\\hline
    9 &             &            &             &           \\\hline
    \end{tabular}
    &
    \begin{tabular}{|r|r|r|r|r|}
      \hline
      
      & \multicolumn{2}{l|}{I2O} & \multicolumn{2}{l|}{I2P} \\\hline
    q & a           & b          & a           & b          \\\hline\hline
    \cellcolor[HTML]{DAE8FC}$0:=\{0,2,4,6,8\}$  & Y            & 0          & $0$ & $1$           \\\hline
    $1:=\{1,3,5,7,9\}$ &             &            &             &            \\\hline
    \end{tabular}
    \end{tabular}
\end{frame}


\begin{frame}
  \frametitle{Konstruktion 2 (extreme)}
  \centering
  \begin{tabular}{cc}
    \begin{tabular}{|l|l|l|l|l|}
      \hline
      & \multicolumn{2}{l|}{I2O} & \multicolumn{2}{l|}{I2P} \\\hline
    q & a           & b          & a           & b          \\\hline\hline
    0  &             &            &             &            \\\hline
    \cellcolor[HTML]{DAE8FC}1 &             &            &             &            \\\hline
    \cellcolor[HTML]{DAE8FC}2 &             &            &             &            \\\hline
    \cellcolor[HTML]{DAE8FC}3 &             &            &             &            \\\hline
    \cellcolor[HTML]{DAE8FC}4 &             &            &             &            \\\hline
    \cellcolor[HTML]{DAE8FC}5 &             &            &             &            \\\hline
    \cellcolor[HTML]{DAE8FC}6 &             &            &             &            \\\hline
    \cellcolor[HTML]{DAE8FC}7 &             &            &             &            \\\hline
    \cellcolor[HTML]{DAE8FC}8 &             &            &             &            \\\hline
    \cellcolor[HTML]{DAE8FC}9 &             &            &             &           \\\hline
    \end{tabular}
    &
    \begin{tabular}{|r|r|r|r|r|}
      \hline
      & \multicolumn{2}{l|}{I2O} & \multicolumn{2}{l|}{I2P} \\\hline
    q & a           & b          & a           & b          \\\hline\hline
    \cellcolor[HTML]{DAE8FC}$0$  &             &            &             &            \\\hline
    $1:=\{1,2,3,4,5,6,7,8,9\}$ &             &            &             &            \\\hline
    \end{tabular}
    \end{tabular}
\end{frame}

\begin{frame}
  \frametitle{$|Abs_{min}|$ -- Anzahl der Tests}
  \footnotesize Ref$_{min} = 100$, $\Sigma_I=4, \Sigma_O=8, m=n$
  \normalsize
\pgfplotsset{width=3.5cm,height=3.5cm}
\begin{center}
    \begin{minipage}{\linewidth}
        \centering

        \begin{tikzpicture}
        \pgfplotsset{compat=newest}
        \definecolor{clr1}{RGB}{250,100,25}
        \definecolor{clr2}{RGB}{100,25,250}
        \definecolor{clr3}{RGB}{25,250,100}
        \definecolor{clr4}{RGB}{150,85,20}
        \definecolor{clr5}{RGB}{85,20,150}
        \definecolor{clr6}{RGB}{20,150,85}
        %\begin{axis}[ legend style={at={(1,1)},xshift=0.5cm,anchor=north west},scale only axis, xmin=3, xmax=20, xlabel={$|\Sigma_I|$}, ylabel={$\#$Test Cases}]%
        \begin{axis}[ legend pos = north west,scale only axis, xmin=2, xmax=45, xlabel={$|Abs_{min}|$}, ylabel={$\#$Tests}]%
            \addplot[mark=square*,mark options={scale=0.3},clr1] table [x=absMin,y=Size_SH1] {data/c1_modulo};
            \label{plot_sh1} \addlegendentry{$\text{s-H}_1$}
            \addplot[mark=square*,mark options={scale=0.3},clr2] table [x=absMin,y=Size_SH2] {data/c1_modulo};
            \label{plot_sh2} \addlegendentry{$\text{s-H}_2$}
            \addplot[mark=square*,mark options={scale=0.3},clr3] table [x=absMin,y=Size_SH3] {data/c1_modulo};
            \label{plot_sh3} \addlegendentry{$\text{s-H}_3$}
        \end{axis}
        \end{tikzpicture}
        \begin{tikzpicture}
            \pgfplotsset{compat=newest}
            \definecolor{clr1}{RGB}{250,100,25}
            \definecolor{clr2}{RGB}{100,25,250}
            \definecolor{clr3}{RGB}{25,250,100}
            \definecolor{clr4}{RGB}{225,85,20}
            \definecolor{clr5}{RGB}{85,20,225}
            \definecolor{clr6}{RGB}{20,225,85}
            %\begin{axis}[ legend style={at={(1,1)},xshift=0.5cm,anchor=north west},scale only axis, xmin=3, xmax=20, xlabel={$|\Sigma_I|$}, ylabel={$\#$Test Cases}]%
            \begin{axis}[scale only axis, xmin=2, xmax=99, xlabel={$|Abs_{min}|$}]%
                \addplot[mark=square*,mark options={scale=0.2},clr1] table [x=absMin,y=Size_SH1] {data/c1_extreme};
                \addplot[mark=square*,mark options={scale=0.2},clr2] table [x=absMin,y=Size_SH2] {data/c1_extreme};
                \addplot[mark=square*,mark options={scale=0.2},clr3] table [x=absMin,y=Size_SH3] {data/c1_extreme};
            \end{axis}
        \end{tikzpicture}
        %}
        \captionof{figure}{Anzahl der Tests in Abhängigkeit von $|Abs_{min}|$, links: modulo-Konstruktion, rechts: extreme-Konstruktion}
    \end{minipage}
    \end{center}
%%%%%%%%%%
\end{frame}

\begin{frame}
  
\pgfplotsset{width=3.5cm,height=3.5cm}
\begin{center}
  \frametitle{$|Abs_{min}|$ -- Anzahl der Testschritte}
  \footnotesize Ref$_{min} = 100$, $\Sigma_I=4, \Sigma_O=8, m=n$
  \normalsize
    \begin{minipage}{\linewidth}
        \centering
        \begin{tikzpicture}
        \pgfplotsset{compat=newest}
        \definecolor{clr1}{RGB}{250,100,25}
        \definecolor{clr2}{RGB}{100,25,250}
        \definecolor{clr3}{RGB}{25,250,100}
        \definecolor{clr4}{RGB}{225,85,20}
        \definecolor{clr5}{RGB}{85,20,225}
        \definecolor{clr6}{RGB}{20,225,85}
        %\begin{axis}[ legend style={at={(1,1)},xshift=0.5cm,anchor=north west},scale only axis, xmin=3, xmax=20, xlabel={$|\Sigma_I|$}, ylabel={$\#$Test Cases}]%
        \begin{axis}[ scale only axis, xmin=2, xmax=99, xlabel={$|Abs_{min}|$}, ylabel={$\#$Testschritte}]%
            \addplot[mark=square*,mark options={scale=0.3},clr1] table [x=absMin,y=Total_SH1] {data/c1_modulo};
            \addplot[mark=square*,mark options={scale=0.3},clr2] table [x=absMin,y=Total_SH2] {data/c1_modulo};
            \addplot[mark=square*,mark options={scale=0.3},clr3] table [x=absMin,y=Total_SH3] {data/c1_modulo};
        \end{axis}
        \end{tikzpicture}
        \begin{tikzpicture}
        \pgfplotsset{compat=newest}
        \definecolor{clr1}{RGB}{250,100,25}
        \definecolor{clr2}{RGB}{100,25,250}
        \definecolor{clr3}{RGB}{25,250,100}
        \definecolor{clr4}{RGB}{225,85,20}
        \definecolor{clr5}{RGB}{85,20,225}
        \definecolor{clr6}{RGB}{20,225,85}
        %\begin{axis}[ legend style={at={(1,1)},xshift=0.5cm,anchor=north west},scale only axis, xmin=3, xmax=20, xlabel={$|\Sigma_I|$}, ylabel={$\#$Test Cases}]%
        \begin{axis}[ legend pos = south east, scale only axis, xmin=2, xmax=99, xlabel={$|Abs_{min}|$}]%
            \addplot[mark=square*,mark options={scale=0.3},clr1] table [x=absMin,y=Total_SH1] {data/c1_extreme};
            \label{plot_sh1} \addlegendentry{$\text{s-H}_1$}
            \addplot[mark=square*,mark options={scale=0.3},clr2] table [x=absMin,y=Total_SH2] {data/c1_extreme};
            \label{plot_sh2} \addlegendentry{$\text{s-H}_2$}
            \addplot[mark=square*,mark options={scale=0.3},clr3] table [x=absMin,y=Total_SH3] {data/c1_extreme};
            \label{plot_sh3} \addlegendentry{$\text{s-H}_3$}
        \end{axis}
        \end{tikzpicture}
        %}
        \captionof{figure}{\label{fig:c1_red_total}Anzahl der Testschritte in Abhängigkeit von $|Abs_{min}|$,
        links: modulo-Konstruktion, rechts: extreme-Konstruktion}
    \end{minipage}
    \end{center}
\end{frame}

\begin{frame}
  \frametitle{$|Abs_{min}|$ -- Testreduktion}
  \footnotesize Ref$_{min} = 100$, $\Sigma_I=4, \Sigma_O=8, m=n$
  \normalsize
  \pgfplotsset{width=3.5cm,height=3.5cm}
  \begin{center}
      \begin{minipage}{\linewidth}
          \centering
          \begin{tikzpicture}
          \pgfplotsset{compat=newest}
          \definecolor{clr1}{RGB}{250,100,25}
          \definecolor{clr2}{RGB}{100,25,250}
          \definecolor{clr3}{RGB}{25,250,100}
          \definecolor{clr4}{RGB}{225,85,20}
          \definecolor{clr5}{RGB}{85,20,225}
          \definecolor{clr6}{RGB}{20,225,85}
          %\begin{axis}[ legend style={at={(1,1)},xshift=0.5cm,anchor=north west},scale only axis, xmin=3, xmax=20, xlabel={$|\Sigma_I|$}, ylabel={$\#$Test Cases}]%
          \begin{axis}[ scale only axis, xmin=2, xmax=35, xlabel={$|Abs_{min}|$}, ylabel={$\mathbf{Red_H\%}$}]%
              \addplot[mark=square*,mark options={scale=0.2},clr1] table [x=absMin,y=Impr_SH1] {data/c1_modulo};
              \addplot[mark=square*,mark options={scale=0.2},clr2] table [x=absMin,y=Impr_SH2] {data/c1_modulo};
              \addplot[mark=square*,mark options={scale=0.2},clr3] table [x=absMin,y=Impr_SH3] {data/c1_modulo};
              
          \end{axis}
          \end{tikzpicture}
          \begin{tikzpicture}
          \pgfplotsset{compat=newest}
          \definecolor{clr1}{RGB}{250,100,25}
          \definecolor{clr2}{RGB}{100,25,250}
          \definecolor{clr3}{RGB}{25,250,100}
          \definecolor{clr4}{RGB}{225,85,20}
          \definecolor{clr5}{RGB}{85,20,225}
          \definecolor{clr6}{RGB}{20,225,85}
          %\begin{axis}[ legend style={at={(1,1)},xshift=0.5cm,anchor=north west},scale only axis, xmin=3, xmax=20, xlabel={$|\Sigma_I|$}, ylabel={$\#$Test Cases}]%
          \begin{axis}[ legend pos = north east, scale only axis, xmin=2, xmax=50, xlabel={$|Abs_{min}|$}]%
              \addplot[mark=square*,mark options={scale=0.2},clr1] table [x=absMin,y=Impr_SH1] {data/c1_extreme};
              \label{plot_sh1} \addlegendentry{$\text{s-H}_1$}
              \addplot[mark=square*,mark options={scale=0.2},clr2] table [x=absMin,y=Impr_SH2] {data/c1_extreme};
              \label{plot_sh2} \addlegendentry{$\text{s-H}_2$}
              \addplot[mark=square*,mark options={scale=0.2},clr3] table [x=absMin,y=Impr_SH3] {data/c1_extreme};
              \label{plot_sh3} \addlegendentry{$\text{s-H}_3$}
          \end{axis}
          \end{tikzpicture}
          %}
          \captionof{figure}{\label{fig:c1_red_percent}Reduktion der Anzahl der Tests in Abhängigkeit von der Anzahl der Zustände des Abstraktionsmodells,
          links: modulo-Konstruktion, rechts: extreme-Konstruktion,
          $m=n$.}
      \end{minipage}
      \end{center}  
\end{frame}

\begin{frame}
  \frametitle{$|Abs_{min}|$ -- Laufzeit}
  \footnotesize Ref$_{min} = 100$, $\Sigma_I=4, \Sigma_O=8, m=n$
  \normalsize
\pgfplotsset{width=3.4cm,height=3.4cm,
yticklabel style={
    /pgf/number format/fixed,
    /pgf/number format/precision=5
},
scaled y ticks=false}

\begin{center}
\begin{minipage}{\linewidth}
    \centering
    \begin{tikzpicture}[baseline=(current axis.south)]
    \pgfplotsset{compat=newest}
    \definecolor{clr1}{RGB}{250,100,25}
    \definecolor{clr2}{RGB}{100,25,250}
    \definecolor{clr3}{RGB}{25,250,100}
    \definecolor{clr4}{RGB}{225,85,20}
    \definecolor{clr5}{RGB}{85,20,225}
    \definecolor{clr6}{RGB}{20,225,85}
    %\begin{axis}[ legend style={at={(1,1)},xshift=0.5cm,anchor=north west},scale only axis, xmin=3, xmax=20, xlabel={$|\Sigma_I|$}, ylabel={$\#$Test Cases}]%
    \begin{axis}[ scale only axis, xmin=2, xmax=99, xlabel={$|Abs_{min}|$}, ylabel={Zeit in ms}]%
      \addplot[mark=square*,mark options={scale=0.1},clr1] table [x=absMin,y=timeSH1] {data/c1_modulo};
      \label{plot_sh1} \addlegendentry{$\text{s-H}_1$}
      \addplot[mark=square*,mark options={scale=0.1},clr2] table [x=absMin,y=timeSH2] {data/c1_modulo};
      \label{plot_sh2} \addlegendentry{$\text{s-H}_2$}
      \addplot[mark=square*,mark options={scale=0.1},clr3] table [x=absMin,y=timeSH3] {data/c1_modulo};
      \label{plot_sh3} \addlegendentry{$\text{s-H}_3$}
    \end{axis}
    \end{tikzpicture}
    \begin{tikzpicture}[baseline=(current axis.south)]
        \pgfplotsset{compat=newest}
        \definecolor{clr1}{RGB}{250,100,25}
        \definecolor{clr2}{RGB}{100,25,250}
        \definecolor{clr3}{RGB}{25,250,100}
        \definecolor{clr4}{RGB}{225,85,20}
        \definecolor{clr5}{RGB}{85,20,225}
        \definecolor{clr6}{RGB}{20,225,85}
        %\begin{axis}[ legend style={at={(1,1)},xshift=0.5cm,anchor=north west},scale only axis, xmin=3, xmax=20, xlabel={$|\Sigma_I|$}, ylabel={$\#$Test Cases}]%
        \begin{axis}[legend pos = north west, scale only axis, xmin=2, xmax=99, xlabel={$|Abs_{min}|$}]%
          \addplot[mark=square*,mark options={scale=0.1},clr1] table [x=absMin,y=timeSH1] {data/c1_extreme};
          \addplot[mark=square*,mark options={scale=0.1},clr2] table [x=absMin,y=timeSH2] {data/c1_extreme};
          \addplot[mark=square*,mark options={scale=0.1},clr3] table [x=absMin,y=timeSH3] {data/c1_extreme};
        \end{axis}
        \end{tikzpicture}
    %}
    \captionof{figure}{\label{fig:c1_speed}Durchschnittliche Zeit in Millisekunden, die für die Testgenerierung in Abhängigkeit von $|Abs_{min}|$ benötigt wurde,
    links: modulo-Konstruktion, rechts: extreme-Konstruktion}
\end{minipage}
\end{center}
\end{frame}

\begin{frame}
  \frametitle{$|Ref_{min}|$ -- Anzahl der Tests}
  \footnotesize $|Abs_{min}| = 4$, $\Sigma_I=4, \Sigma_O=8, m=n$
  \normalsize
\pgfplotsset{width=3.5cm,height=3.5cm}
\begin{center}
    \begin{minipage}{\linewidth}
        \centering
        \begin{tikzpicture}
        \pgfplotsset{compat=newest}
        \definecolor{clr1}{RGB}{250,100,25}
        \definecolor{clr2}{RGB}{100,25,250}
        \definecolor{clr3}{RGB}{25,250,100}
        \definecolor{clr4}{RGB}{150,85,20}
        \definecolor{clr5}{RGB}{85,20,150}
        \definecolor{clr6}{RGB}{20,150,85}
        %\begin{axis}[ legend style={at={(1,1)},xshift=0.5cm,anchor=north west},scale only axis, xmin=3, xmax=20, xlabel={$|\Sigma_I|$}, ylabel={$\#$Test Cases}]%
        \begin{axis}[ legend pos = north west,scale only axis, xmin=10, xmax=100, xlabel={$|Ref_{min}|$}, ylabel={$\#$Tests}]%
            \addplot[mark=square*,mark options={scale=0.2},clr1] table [x=refMin,y=Size_SH1] {data/c2_modulo};
            \label{plot_sh1} \addlegendentry{$\text{s-H}_1$}
            \addplot[mark=square*,mark options={scale=0.2},clr2] table [x=refMin,y=Size_SH2] {data/c2_modulo};
            \label{plot_sh2} \addlegendentry{$\text{s-H}_2$}
            \addplot[mark=square*,mark options={scale=0.2},clr3] table [x=refMin,y=Size_SH3] {data/c2_modulo};
            \label{plot_sh3} \addlegendentry{$\text{s-H}_3$}
        \end{axis}
        \end{tikzpicture}
        \begin{tikzpicture}
            \pgfplotsset{compat=newest}
            \definecolor{clr1}{RGB}{250,100,25}
            \definecolor{clr2}{RGB}{100,25,250}
            \definecolor{clr3}{RGB}{25,250,100}
            \definecolor{clr4}{RGB}{225,85,20}
            \definecolor{clr5}{RGB}{85,20,225}
            \definecolor{clr6}{RGB}{20,225,85}
            %\begin{axis}[ legend style={at={(1,1)},xshift=0.5cm,anchor=north west},scale only axis, xmin=3, xmax=20, xlabel={$|\Sigma_I|$}, ylabel={$\#$Test Cases}]%
            \begin{axis}[scale only axis, xmin=10, xmax=100, xlabel={$|Ref_{min}|$}]%
                \addplot[mark=square*,mark options={scale=0.2},clr1] table [x=refMin,y=Size_SH1] {data/c2_extreme};
                \addplot[mark=square*,mark options={scale=0.2},clr2] table [x=refMin,y=Size_SH2] {data/c2_extreme};
                \addplot[mark=square*,mark options={scale=0.2},clr3] table [x=refMin,y=Size_SH3] {data/c2_extreme};
            \end{axis}
        \end{tikzpicture}
        %}
        \captionof{figure}{Anzahl der Tests in Abhängigkeit von der Anzahl der Zustände des Referenzmodells,
        links: modulo-Konstruktion, rechts: extreme-Konstruktion}
    \end{minipage}
    \end{center}
\end{frame}

\begin{frame}
  \frametitle{$|Ref_{min}|$ -- Testreduktion}
  \footnotesize $|Abs_{min}| = 4$, $\Sigma_I=4, \Sigma_O=8, m=n$
  \normalsize
\pgfplotsset{width=3.5cm,height=3.5cm}
\begin{center}
    \begin{minipage}{\linewidth}
        \centering
        \begin{tikzpicture}
        \pgfplotsset{compat=newest}
        \definecolor{clr1}{RGB}{250,100,25}
        \definecolor{clr2}{RGB}{100,25,250}
        \definecolor{clr3}{RGB}{25,250,100}
        \definecolor{clr4}{RGB}{225,85,20}
        \definecolor{clr5}{RGB}{85,20,225}
        \definecolor{clr6}{RGB}{20,225,85}
        %\begin{axis}[ legend style={at={(1,1)},xshift=0.5cm,anchor=north west},scale only axis, xmin=3, xmax=20, xlabel={$|\Sigma_I|$}, ylabel={$\#$Test Cases}]%
        \begin{axis}[  scale only axis, ymin=0,ymax=22,xmin=10, xmax=100, xlabel={$|Ref_{min}|$}, ylabel={$\mathbf{Red_H\%}$}]%
            \addplot[mark=square*,mark options={scale=0.3},clr1] table [x=refMin,y=Impr_SH1] {data/c2_modulo};
            
            \addplot[mark=square*,mark options={scale=0.3},clr2] table [x=refMin,y=Impr_SH2] {data/c2_modulo};
            
            \addplot[mark=square*,mark options={scale=0.3},clr3] table [x=refMin,y=Impr_SH3] {data/c2_modulo};
            
            
        \end{axis}
        \end{tikzpicture}
        \begin{tikzpicture}
        \pgfplotsset{compat=newest}
        \definecolor{clr1}{RGB}{250,100,25}
        \definecolor{clr2}{RGB}{100,25,250}
        \definecolor{clr3}{RGB}{25,250,100}
        \definecolor{clr4}{RGB}{225,85,20}
        \definecolor{clr5}{RGB}{85,20,225}
        \definecolor{clr6}{RGB}{20,225,85}
        %\begin{axis}[ legend style={at={(1,1)},xshift=0.5cm,anchor=north west},scale only axis, xmin=3, xmax=20, xlabel={$|\Sigma_I|$}, ylabel={$\#$Test Cases}]%
        \begin{axis}[ legend pos = south east, scale only axis, ymin=0,ymax=45,xmin=10, xmax=100, xlabel={$|Ref_{min}|$}]%
            \addplot[mark=square*,mark options={scale=0.3},clr1] table [x=refMin,y=Impr_SH1] {data/c2_extreme};
            \label{plot_sh1} \addlegendentry{$\text{s-H}_1$}
            \addplot[mark=square*,mark options={scale=0.3},clr2] table [x=refMin,y=Impr_SH2] {data/c2_extreme};
            \label{plot_sh2} \addlegendentry{$\text{s-H}_2$}
            \addplot[mark=square*,mark options={scale=0.3},clr3] table [x=refMin,y=Impr_SH3] {data/c2_extreme};
            \label{plot_sh3} \addlegendentry{$\text{s-H}_3$}
        \end{axis}
        \end{tikzpicture}
        %}
        \captionof{figure}{\label{fig:c2_red_percent}Reduktion der Anzahl der Tests in Abhängigkeit von der Anzahl der Zustände des Referenzmodells,
        links: modulo-Konstruktion, rechts: extreme-Konstruktion}
    \end{minipage}
    \end{center}
\end{frame}